\begin{abstract}

Both in theoretical and applied research of machine translation it is often assumed that translation between natural languages can be treated in a compositional fashion, but it has proven far from trivial to develop a compositional translation system, or theoretically show it exists. In this thesis, an empirical investigation of compositionality of translation is presented, of which the main purpose is to find empirical evidence for the compositionality of actual translation data in the form of parallel corpora.

All maximally compositional translation structures of sentences in parallel corpora aligned at the word level were studied, to gain information about the system that generated them. In particular, it was studied whether monolingual information from dependency parses could be the basis of this underlying system.

Experiments showed that hardly over fifty percent of the dependency relations were preserved during translation if no modifications in the dependency relations were allowed. Considering deeper versions of dependency parses boosted this score with over thirty percentage points for all datasets. A manual analysis showed that most of the structure deviations were caused by errors in the data or systematic differences between the languages. 

The results are encouraging for pursuing development of compositional translation systems based on dependency parses. A proposal for doing so is presented in the discussion of this thesis. Tools to execute this proposal, as well as tools to conduct further empirical research, have been made available.
\end{abstract}